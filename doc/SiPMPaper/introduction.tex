\section{Introduction}

Scintillating materials are widely used in detectors of ionizing radiation. 
They a very common as primary sensor in calorimetric application, either serving simultaneously as an absorber material, such as solid crystals of plastic scintillators, or in combination with passiv absorbers in a layered arrangement, referred to as sampling calorimeters. Scintillators are also used to detected single charged particles where they achieve a very high efficiency with a thin enough layer of scintillator as to not impact the particle trajectory significantly.
To convert the primary scintillation light signal into an electrical signal a photo detector is coupled to the scintiallator volume, either directly or via a light guiding structure.
Silicon Photo Multipliers (SiPM) are a common choice as a photo detector in todays applications.
In this paper we present studies on SiPMs and scintillator based detectors with SiPM readout wiht a focus on precise timing measurements for ionizing radiation. 
%
% If we want the HL-LHC reference ...
%
Precise time of arrival measurements are draw much attention in context of detector R\&D for the high luminosity upgrade of the Large Hadron Collider (HL-LHC) as well as for future high energy hadron colliders.  
These hadron colliders must provide large 
instantaneous luminosity well above $10^{35}$~$\mathrm{cm}^{-2}\mathrm{s}^{-1}$.
With current accelerator and particle detector capabilities, such a high 
instantaneous luminosity will result in very large amounts
of pileup, exceeding several hundreds of simultaneous inelastic collisions per
bunch crossing. Therefore, the crucial ability to identify the origin 
of the particles produced at the different interaction points will be severely 
degraded. Precision timing detectors can be used to recover the ability to 
discriminate between particles produced by different inelastic collisions \cite{adielba}.
For beam bunch profiles similar to that of the LHC, a detector 
that can measure the time of arrival of all final state particles in a collision
with a precision of $20-30$~ps can effectively reduce the impact of
pileup by a factor of $5$ to $10$. 
%
% 
% 
With a timing resolution of below $10$~ps for individual charged particles, the equivalent spatial 
resolution becomes relevant for the track reconstruction \cite{4dtracking} for hadron collider experiments.
A timing resolution in the ps range for photons is also very interesting for optical Time Projection Chambers which are discussed for large volume detectors in neutrino experiments \cite{lappd, otpc} 
%
%
%

