\section{Introduction}
\label{sec:introduction}

%Scintillating materials are widely used in detectors of ionizing radiation. They
%are very common as primary sensors in calorimetric applications, either serving
%simultaneously as an absorber material, such as solid crystals, or in
%combination with passive absorbers in a layered arrangement, referred to as
%sampling calorimeters. Scintillators are also used to detect charged particles
%where they achieve a very high efficiency with a layer of thickness of about a
%few millimeters. LYSO has been proposed as a scintillating crystal for future
%calorimeters due to its large scintillation light yield~\cite{LYSOProperties,
%Yang:2015nsa}. To convert the primary scintillation light signal into an
%electrical signal a photodetector is coupled to the scintillator volume, either
%directly or via a light guiding structure. Silicon Photo Multipliers (SiPM) are
%a common choice as a photodetector in contemporary applications. In this paper
%we present studies on SiPMs and a LYSO-based sampling calorimeter with
%SiPM readout with a focus on precise timing measurements. 

Precise time of arrival measurements have recently drawn much attention in the
context of detector R\&D for the high luminosity upgrade of the Large Hadron
Collider (HL-LHC) as well as for future high energy hadron colliders. These
hadron colliders must provide large instantaneous luminosity well above
$10^{35}$~$\mathrm{cm}^{-2}\mathrm{s}^{-1}$. With current accelerator and
particle detector capabilities, such a high instantaneous luminosity will result
in very large amounts of simultaneous particle collisions (pileup) exceeding
several hundreds per bunch crossing. Therefore, the crucial ability to identify
the origin of the particles produced at the different interaction points will be
severely degraded. Precision timing detectors can be used to recover the ability
to discriminate between particles produced by different inelastic collisions
\cite{adielba}. For colliding particle beams with time spread of the order of
$150--200$~ps, as projected for the HL-LHC, a detector that can measure the time of
arrival of particles can identify and reject particles from pileup collisions
based on their time of arrival. Therefore with a timing detector with a timing
precision of $20--30$~ps, the number of pileup collisions which cannot be rejected
based on their time of arrival will be about $20$ to $40$ and is similar to
operational conditions in Run 2 of the LHC. A timing
resolution in the range of $30-50$~ps is also very interesting for
optical time projection chambers which are discussed for large volume detectors
in neutrino experiments and neutrino-less double-beta
decay~\cite{Aberle:2013jba, otpc}. 

In our previous work in Ref.~\cite{Anderson:2015gha}, we demonstrated the
feasibility of achieving $30$~ps resolution for electromagnetic showers using a
sampling calorimeter based on LYSO crystal scintillators. Using micro-channel
plate photomultipliers (MCP-PMTs) to read out photons on the edge of each LYSO
layer, we achieved a time resolution of $55$~ps for electrons with $32$~GeV of
energy. Using wavelength-shifting fibers to extract the light into the MCP-PMTs,
we achieved a time resolution close to $100$~ps. We concluded that the goal of
$30$~ps time resolution was within reach provided that we can realize similar
performance using more economical photodetectors, extract the light using means
that are radiation-hard, and achieve improved light collection efficiency. In
this paper, we report on updated studies that demonstrate the time resolution
performance using silicon photomultiplier (SiPM) detectors that are more
economically scalable to the size of modern collider experiments, and 
radiation-hard wavelength-shifting quartz capillaries 
that can maintain its transparency under the harsh radiation conditions of the HL-LHC. 

The paper is organized as follows. In Sec.~\ref{sec:sipm} we give a brief
overview of the SiPM sensors in the context of our research. In
Sec.~\ref{sec:setup} we describe the experimental techniques we employ in our
precision timing measurements as well as the specific setups we used for the
studies presented in this paper. In Sec.~\ref{sec:beamtiming} we present the
results of timing measurements using SiPMs as photodetectors to read out
scintillation light from LYSO crystals exposed to electrons in the GeV energy
range. In Sec.~\ref{sec:lasertiming} we evaluate the impact of the intrinsic
timing performance of SiPM devices on the calorimeter time measurement by
measuring the time resolution for SiPMs injected with light from a fast laser.
