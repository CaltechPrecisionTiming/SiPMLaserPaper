\section{Introduction}
\label{sec:introduction}

Scintillating materials are widely used in detectors of ionizing radiation. They
are very common as primary sensors in calorimetric applications, either serving
simultaneously as an absorber material, such as solid crystals of plastic
scintillators, or in combination with passive absorbers in a layered
arrangement, referred to as sampling calorimeters. Scintillators are also used
to detected single charged particles where they achieve a very high efficiency
with a thin layer with thickness on the order of one mm that does not significantly
impact the particle trajectory for particles with momentum on the order of a GeV. To
convert the primary scintillation light signal into an electrical signal a photo
detector is coupled to the scintillator volume, either directly or via a light
guiding structure. Silicon Photo Multipliers (SiPM) are a common choice as a
photo detector in contemporary applications. In this paper we present studies on
SiPMs and scintillator-based detectors with SiPM readout with a focus on precise
timing measurements for ionizing radiation. 

Precise time of arrival measurements have drawn much recent attention in the 
context of detector R\&D for the high luminosity upgrade of the Large Hadron 
Collider (HL-LHC) as well as for future high energy hadron colliders.  
These hadron colliders must provide large 
instantaneous luminosity well above $10^{35}$~$\mathrm{cm}^{-2}\mathrm{s}^{-1}$.
With current accelerator and particle detector capabilities, such a high 
instantaneous luminosity will result in very large amounts
of simultaneous particle collisions (pileup) exceeding several hundreds per
bunch crossing. Therefore, the crucial ability to identify the origin 
of the particles produced at the different interaction points will be severely 
degraded. Precision timing detectors can be used to recover the ability to 
discriminate between particles produced by different inelastic collisions \cite{adielba}.
For particle beams with bunch profiles similar to that of the LHC, a detector 
that can measure the time of arrival of all final state particles 
with a precision of $20-30$~ps can effectively reduce the impact of
pileup by a factor of $5$ to $10$. 

With a timing resolution below $10$~ps for individual charged particles, the
equivalent spatial resolution is sufficiently good to improve the performance of
track reconstruction \cite{Neri:2016bng} for hadron collider experiments. A timing
resolution in the picosecond range for photons is also very interesting for optical time
projection chambers which are discussed for large volume detectors in neutrino
experiments and neutrino-less double-beta decay~\cite{Aberle:2013jba, otpc}. 

In this paper we present studies of the timing performance of a
LYSO-based sampling calorimeter read out via four wavelength-shifting fibers
optically coupled to silicon photomultipliers (SiPMs). 
In Sec.~\ref{sec:sipm} we give a brief overview of the SiPM sensors in the
context of our research. In Sec.~\ref{sec:setup} we describe the experimental
techniques we employ in our precision timing measurements as well as the
specific setups we used for the studies presented in this paper. 
In Sec.~\ref{sec:beamtiming} we present the results of timing measurements using 
SiPMs as photo-detectors to read out scintillation light from LYSO crystals exposed 
to electrons in the GeV energy range. In Sec.~\ref{sec:lasertiming} we evaluate the
impact of the intrinsic timing performance of SiPM devices on the calorimeter 
time measurement by measuring the time resolution for SiPMs injected with
light from a fast laser.
