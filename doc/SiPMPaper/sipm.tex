\section{SiPM}
\label{sec:sipm}

Silicon Photomultipliers (SiPM) are pixelated photodetectors that are widely
used in contemporary high-energy physics experiments. Their compactness and form
factor make them ideal for many applications including calorimeters and charged
particle detectors. They are also widely used for positron emission tomography
(PET) detectors together with LYSO scintillating crystals for medical imaging
purposes~\cite{Vandenberghe2016}, where new studies have improved the 
timing resolution below $100$~ps~\cite{LecoqTOFPET} and 
 can yield substantial improvements in image resolution and imaging 
 capabilities. The size of each SiPM device typically
ranges between $1\times 1$~$\mathrm{mm}^{2}$ and $6\times 6$~$\mathrm{mm}^{2}$,
with the size of each pixel ranging between $10\mu$m to $50\mu$m. SiPMs operate
at relatively high gain between $10^{5}$ and $10^{6}$, and have single photon detection
efficiency ranging from $10\%$ to $50\%$. 
%As each pixel operates in geiger mode, it is essentially a digital
%device. More than one photon impinging on a single pixel yields the same signal
%as a single photon impinging on that pixel. Therefore as the number of photons
%approaches the total number of pixels, the SiPM experiences a slow saturation as
%the signal response slowly becomes non-linear. Total saturation occurs 
i%f the number of photons exceed the number of pixels in the device. 


SiPMs have a typical thermal dark count rate of about
$0.1$~MHz/$\mathrm{mm}^{2}$ at room temperature, which can be strongly decreased
when operated at lower temperatures. Typical operational temperatures range from
$20$ to $30$ degrees Celsius, but can be as low as $-30$ degrees Celsius. SiPMs
have been tested for the impact of radiation damage up to an equivalent neutron
rate of $2\times10^{14}$~$\mathrm{cm}^{2}$, and its performance have been shown
to be robust when operated at temperatures below $5$~degrees
Celsius.~\cite{SiPMIrradiated1,SiPMIrradiated2}. However, when operated at the
same temperature, the thermal dark count rate increases significantly with large
irradiation. The SiPMs used for our studies are Hamamatsu 
MPPC S12571-010P and S12571-015P both of 
size $1\times1\mathrm{mm}^{2}$, and S12572-15C and S12572-25C both of size 
$3\times3\mathrm{mm}^{2}$. These SiPMs are chosen to allow us to study
the impact of the size of the sensitive area and the size of individual pixels 
on the timing performance. Studies of the impact of SiPMs from alternative 
manufacturers are left for future work. Some relevant details of the SiPM parameters
are summarized in Table~\ref{tab:SiPMParameters} below.

\begin{table}[!ht]
\begin{center}
\caption{Summary of performance parameters for SiPMs used in our studies.}
\label{tab:SiPMParameters}
\begin{tabular}{|c|c|c|c|c|}
\hline
Parameter & S12571-010C & S12571-015C & S12572-15C & S12572-25C\\
\hline
Photosensitive area              & $1\times1\mathrm{mm}^{2}$ & $1\times1\mathrm{mm}^{2}$ & $3\times3\mathrm{mm}^{2}$ & $3\times3\mathrm{mm}^{2}$\\
Pixel Pitch                            & $10\mu$m                             & $15\mu$m                             & $15\mu$m                              & $25\mu$m\\
Number of Pixels                 &  10000                                    &  4489                                      & 40000                                     & 14400                       \\
Dark Count Rate                  &  $0.1$~MHz                            &  $0.1$~MHz                            & $1$~MHz                                 & $1$~MHz                  \\
Gain                                    & $1.35\times10^{5}$                  & $2.3\times10^{5}$                & $2.3\times10^{5}$                 & $5.15\times10^{5}$  \\
Terminal Capacitance          & $35$~pF                                 & $35$~pF                                  & $320$~pF                                & $320$~pF                  \\
Spectral Response Range     & $320$-$900$~nm                  & $320$-$900$~nm                  & $320$-$900$~nm                   & $320$-$900$~nm     \\
Peak Sensitivity Wavelength & $470$~nm                              & $460$~nm                              & $460$~nm                               & $450$~nm               \\         
\hline
\end{tabular}
\end{center}
\end{table}