\section{Setup and Experimental Apparatus }
\label{sec:setup}

We performed measurements of SiPM properties in the laboratory using
signals from a class 3R PiLas laser which produces light at a wavelength of
$407$~nm. Beam measurements were performed at the H4 beam-line of the CERN
North-Area test-beam facility, which provides secondary beams of energies
ranging between $20$~GeV and $400$~GeV. Electrons can be provided from tertiary beams up to 
$250$~GeV with acceptable efficiency. The beams are composed of a mixture of
electrons and pions. The electron fraction in the beam is typically larger than 
$75\%$ and close to $100\%$ for beam energy above $100$~GeV.

The data acquisition (DAQ) system uses a CAEN V1742 switched capacitor digitizer
based on the DRS4 chip~\cite{DRS4}, whose electronic time resolution has been
measured to be $4$~ps. Data readout for the laser-based measurements are
triggered by an external digital trigger signal, while at the H4 beamline
readout is triggered by a signal in a photomultiplier tube coupled to a
$3$~$\mathrm{cm}$~$\times$~$3$~$\mathrm{cm}$ plastic scintillator located about
one meter upstream from our detectors. A micro-channel plate photo-multiplier
(MCP-PMT) detector is used to provide a very precise reference time-stamp in
order to measure the time resolution of the SiPM signals.

\subsection{Setup for Laser-based SiPM Timing Measurements}

SiPMs are mounted on a printed circuit board (PCB) with the circuit shown in
Figure~\ref{fig:Circuit}. The high-pass filter is used to decrease the rise-time
of signal pulses by removing low-frequency components of the signal pulse. As a
consequence, the resulting signal pulse will be smaller in amplitude and will
have a faster rise-time. The SiPMs are mechanically attached to an optical
breadboard enclosed within a box lined with copper foil for RF shielding. The
laser is injected via a light guide fiber mounted on an optical holder. The
laser beam is immediately split by a 50/50 beam splitter and half of the light
is directed onto the MCP-PMT while the other half of the light is directed onto
the SiPM under test. A photograph of the setup is shown in
Figure~\ref{fig:laserSetup}. The Photek-240 MCP-PMT is used as the reference
time detector whose time resolution has been measured to be below $7$~ps for
beam particles~\cite{MCPShowerMaxPaper}. To cover a large range of laser beam
intensity, neutral density (ND) filters with ND number between 0.2 and 2.4 are
placed between the beam splitter and the SiPM under test.

\begin{figure}[htbp]
\centering
\includegraphics[width=0.60\textwidth]{figures/CircuitDiagramNew.pdf}
\caption{A schematic diagram of the circuit used to read out the SiPMs. The SiPM is labeled as MPPC in the diagram. }
\label{fig:Circuit}
\end{figure}



%Figure: Diagram of detector elements
\begin{figure}[htbp] 
\centering
\includegraphics[width=0.70\textwidth]{figures/SiPMSetup1.pdf} 
\caption{Photograph of the Laser-based SiPM timing measurement setup.} 
\label{fig:laserSetup} 
\end{figure} 

\subsection{Setup for Timing Measurements of Scintillators with SiPM readout.}
\label{sec:LYSOSiPMSetup}
The experimental setup we use for the calorimetric timing measurements 
is shown diagrammatically in Fig.~\ref{fig:TestbeamSchematic} and consists
of a single cell of a sampling calorimeter with 29 alternating layers of LYSO
crystal and tungsten absorber, known as a Shashlik sampling
calorimeter configuration. The lateral dimensions are
$14\times14$~$\mathrm{mm}^{2}$. The total depth of the cell is about $11.5$~cm
with the LYSO layers having a thickness of $1.5$~mm. The same cell has been used
to measure the timing performance in comparison to the timing performance of a
single monolithic crystal of LYSO~\cite{Anderson:2015gha}. A scintillator
counter of size $1\times 1$~$\mathrm{cm}^{2}$, mounted close to the calorimeter
cell, is used to select events impinging on the center of the calorimeter cell. The
scintillation light from the LYSO plates is extracted with four
wavelength-shifting (WLS) fibers with $1$~mm diameter. The fibers are coupled to four different types
of Hamamatsu SiPMs with 10, 15 and 25~$\mu$m pixel size and $1\times
1$~$\mathrm{mm}^{2}$ and $3\times 3$~$\mathrm{mm}^{2}$ sensor
size~\cite{hamamatsuMPPC}. SiPMs are mounted on a printed circuit board (PCB) with the 
circuit shown in Figure~\ref{fig:Circuit}. The high-pass filter is used to decrease the rise-time of signal 
pulses by removing low-frequency components of the signal pulse. As a consequence, the resulting 
signal pulse will be smaller in amplitude and will have a faster rise-time.
We do not amplify the output signal of the SiPMs, only exploiting the very large
light yield of the LYSO scintillator and the intrinsic amplification of the
SiPMs. All four SiPMs were operated at a bias voltage of $70$~V, and due to the 
variance of breakdown voltages of the different SiPMs the operational gain was 
different for each channel. A labeled photograph of the setup in the H4 beamline is shown in
Figure~\ref{fig:TestbeamSetup}. More details of the calorimetric
performance of the Shashlik configuration are discussed in
references~\cite{shashlik1}~and~\cite{shashlik2}.

%Figure: Diagram of detector elements
\begin{figure}[htbp] 
\centering
\includegraphics[width=0.70\textwidth]{figures/ShashlikFiberSetupSchematic} 
\caption{Schematic diagram of the testbeam experimental setup.} 
\label{fig:TestbeamSchematic} 
\end{figure} 

%Figure: Diagram of detector elements
\begin{figure}[htbp] 
\centering
\includegraphics[width=0.70\textwidth]{figures/ShashlikTBSetupDiagram} 
\caption{Photograph of the timing measurement setup in the H4 beamline.} 
\label{fig:TestbeamSetup} 
\end{figure} 

In addition to plastic WLS fibers we also tested quartz capillaries filled with
liquid wavelength shifter using DSB as a wavelength-shifting
agent~\cite{Baumbaugh:2016vcg}. To optically couple the quartz capillaries to the
SiPMs we use a clear plastic fiber light guide which is connected to the end of
the quartz capillary with a metal sleeve tube. The same clear fiber coupler is
used for the plastic WLS fibers to maintain equivalent light collection
efficiency. The ratio of the light collection efficiency between the plastic
fibers and the quartz capillaries approximately scales with the ratio of the
diameter of the plastic fiber and the liquid core of the quartz capillary. This
ratio is about $3$ for the fibers and capillaries we used.

The Photek 240 MCP-PMT, used as the reference time detector, is placed behind the
calorimeter cell and detects secondary shower particles escaping from the
Shashlik calorimeter cell as we did in our previous studies \cite{Anderson:2015gha}.
The time resolution is extracted by measuring the time difference between the
reference counter and the calorimeter cell over an ensemble of shower events.
The time stamp for the reference counter and calorimeter cell is extracted
from a Gaussian fit to the peak and a linear fit to the rising edge, 
respectively, as described below in Section~\ref{sec:reco}.

We measure the timing performance of the calorimeter cell with high energy electron beams
in a range between 20 GeV and 200 GeV. Selection cuts on the signal amplitude in the Shashlik cell
suppresses the pion contamination in the beam. The impact point of the electrons
onto the calorimeter cell is measured with a fiber hodoscope with a precision of
better than $1$~mm. As timing measurements are affected by shower containment,
we restrict the time measurements to events where shower containment is large.
This is achieved by using events where the beam particle impacts in the center
of the calorimeter cell within a restricted area between
$2\times2$~$\mathrm{mm}^{2}$ and $6\times8$~$\mathrm{mm}^{2}$ depending on the
exact setup. 

\subsection{Timestamp Reconstruction} \label{sec:reco} The time-stamp for all
signals is reconstructed by fitting the pulse waveform with an appropriate
functional form. Signal pulses from the MCP-PMTs as well as direct laser light
signals on the SiPMs exhibit a very fast rise and decay. Therefore, we fit a
Gaussian function to a $1.4$~ns window around the peak of the pulse and extract
the time-stamp as the mean parameter of the Gaussian function. Scintillation
signal pulses read out by the SiPM sensors have a much longer decay time. For
these signals, we fit a linear function to time sample points between $10\%$ and
$60\%$ of the pulse maximum and the time-stamp is assigned as the time at which
the fitted linear function rises to $20\%$ of the pulse maximum. More details of
the time-stamp reconstruction can be found in reference~\cite{Anderson:2015gha}.
